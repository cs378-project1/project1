\documentclass[12pt]{article}

\usepackage{amssymb}
\usepackage{ifthen}
\usepackage[table]{xcolor}
\usepackage{minitoc}
\usepackage{array}

\definecolor{yellow}{cmyk}{0,0,1,0}
\renewcommand{\arraystretch}{1.4}
\newcommand{\R}{\mathbb{R}}

\usepackage{colortbl}

% Page size
\setlength{\oddsidemargin}{-0.5in}
\setlength{\evensidemargin}{-0.5in}
\setlength{\textheight}{10.25in}
\setlength{\textwidth}{7.0in}
\setlength{\topmargin}{-1.35in}

\renewcommand{\arraycolsep}{3pt}


\input color_flatex

\begin{document}
\pagestyle{empty}


\resetsteps % reset all definitions

% Insert output from Spark webpage below


\resetsteps      % Reset all the commands to create a blank worksheet  

% Define the operation to be computed


\resetsteps      % Reset all the commands to create a blank worksheet  

% Define the operation to be computed

\renewcommand{\operation}{ \left[ C \right] := \mbox{\sc syr2k\_ln\_unb\_var9}( A, B, C ) }

\renewcommand{\routinename}{\operation}

% Step 1a: Precondition 

\renewcommand{\precondition}{
	C = \widehat{C}
}

% Step 1b: Postcondition 

\renewcommand{\postcondition}{ 
	\left[C \right]
	=
	\mbox{syr2k\_ln}( A, B, \widehat{C} )
}

% Step 2: Invariant 

\renewcommand{\invariant}{
	C =
	{A_L} {B_L^T} + {B_L} {A_L^T} +  {\widehat{C}}
}

% Step 3: Loop-guard 

\renewcommand{\guard}{
	n( A_L ) < n( A )
}

% Step 4: Initialize 

\renewcommand{\partitionings}{
	$
	A 
	\rightarrow
	\left(\begin{array}{c I c}
	{A_L} & {A_R}
	\end{array}\right)
	$
	,
	$
	B 
	\rightarrow
	\left(\begin{array}{c I c}
	{B_L} & {B_R}
	\end{array}\right)
	$
}

\renewcommand{\partitionsizes}{
	$ A_L $ has $ 0 $ columns,
	$ B_L $ has $ 0 $ columns
}

% Step 5a: Repartition the operands 

\renewcommand{\repartitionings}{
	$  
	\left(\begin{array}{c I c}
	{A_L} & {A_R}
	\end{array}\right)
	\rightarrow  
	\left(\begin{array}{c I c c}
	{A_0} & {a_1} & {A_2}
	\end{array}\right)
	$
	,
	$  
	\left(\begin{array}{c I c}
	{B_L} & {B_R}
	\end{array}\right)
	\rightarrow  
	\left(\begin{array}{c I c c}
	{B_0} & {b_1} & {B_2}
	\end{array}\right)
	$
}

\renewcommand{\repartitionsizes}{
	$ a_1 $ has $ 1 $ column,
	$ b_1 $ has $ 1 $ column
	}

% Step 5b: Move the double lines 

\renewcommand{\moveboundaries}{
	$  
	\left(\begin{array}{c I c}
	{A_L} & {A_R}
	\end{array}\right)
	\leftarrow  
	\left(\begin{array}{c c I c}
	{A_0} & {a_1} & {A_2}
	\end{array}\right)
	$
	,
	$  
	\left(\begin{array}{c I c}
	{B_L} & {B_R}
	\end{array}\right)
	\leftarrow  
	\left(\begin{array}{c c I c}
	{B_0} & {b_1} & {B_2}
	\end{array}\right)
	$
}


% Step 6: State after repartitioning

\renewcommand{\beforeupdate}{
	$ 
	C
	=
	{A_0} {B_0}^T + {B_0} {A_0}^T + \widehat{C}
	$
}

% Step 7: State after moving of double lines
% Note: The below needs editing!!!

\renewcommand{\afterupdate}{
	$  
	C
	=
	{A_0} {B_0}^T + {B_0} {A_0}^T + {a_1} {b_1^T} + {b_1} {a_1^T} + \widehat{C}
	$
}

% Step 8: Insert the updates required to change the 
%         state from that given in Step 6 to that given in Step 7
% Note: The below needs editing!!!

\renewcommand{\update}{
	$
	C := {a_1} {b_1^T} + {b_1} {a_1^T} + C
	$
}








\begin{center}
	\FlaWorksheet
\end{center}

\newpage

\begin{figure}[p]
\begin{center}
	\FlaWorksheetZero
\end{center}
\end{figure}

\begin{figure}[p]
\begin{center}
	\FlaWorksheetOne
\end{center}
\end{figure}

\begin{figure}[p]
\begin{center}
	\FlaWorksheetTwo
\end{center}
\end{figure}

\begin{figure}[p]
\begin{center}
	\FlaWorksheetThree
\end{center}
\end{figure}

\begin{figure}[p]
	\begin{center}
	\FlaWorksheetFour
\end{center}
\end{figure}

\begin{figure}[p]
	\begin{center}
	\FlaWorksheetFive
\end{center}
\end{figure}

\begin{figure}[p]
	\begin{center}
	\FlaWorksheetSix
\end{center}
\end{figure}

\begin{figure}[p]
	\begin{center}
	\FlaWorksheetSeven
\end{center}
\end{figure}

\begin{figure}[p]
	\begin{center}
	\FlaWorksheetEight
\end{center}
\end{figure}

\begin{figure}[p]
	\begin{center}
	\FlaWorksheetNine
\end{center}
\end{figure}

\begin{figure}[p]
\begin{center}
	\FlaAlgorithm
\end{center}
\end{figure}

\end{document}