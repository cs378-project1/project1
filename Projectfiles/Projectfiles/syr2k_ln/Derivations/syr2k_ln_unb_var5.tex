\documentclass[12pt]{article}

\usepackage{amssymb}
\usepackage{ifthen}
\usepackage[table]{xcolor}
\usepackage{minitoc}
\usepackage{array}

\definecolor{yellow}{cmyk}{0,0,1,0}
\renewcommand{\arraystretch}{1.4}
\newcommand{\R}{\mathbb{R}}

\usepackage{colortbl}

% Page size
\setlength{\oddsidemargin}{-0.5in}
\setlength{\evensidemargin}{-0.5in}
\setlength{\textheight}{10.25in}
\setlength{\textwidth}{7.0in}
\setlength{\topmargin}{-1.35in}

\renewcommand{\arraycolsep}{3pt}


\input color_flatex

\begin{document}
\pagestyle{empty}


\resetsteps % reset all definitions

% Insert output from Spark webpage below


\resetsteps      % Reset all the commands to create a blank worksheet  

% Define the operation to be computed


\resetsteps      % Reset all the commands to create a blank worksheet  

% Define the operation to be computed

\renewcommand{\operation}{ C := A B ^ T + B A ^ T + C }

\renewcommand{\routinename}{\operation}

% Step 1a: Precondition 

\renewcommand{\precondition}{
	C = \widehat C
}

% Step 1b: Postcondition 

\renewcommand{\postcondition}{ 
	C := A B ^ T + B A ^ T + \widehat C.
}

% Step 2: Invariant 
% Note: Right-hand side of equalities must be updated appropriately

\renewcommand{\invariant}{
	\left(\begin{array}{c I c}
		C_{TL} & * \\ \whline
		C_{BL} & C_{BR}
		\end{array}\right)
		 = 
		 \left(\begin{array}{c I c}
			\widehat C_{TL} & * \\ \whline
			A_B B_T^T + \widehat C_{BL} & A_B B_B^T + B_B A_B^T + \widehat C_{BR}
			\end{array}\right)
}

% Step 3: Loop-guard 

\renewcommand{\guard}{
	m( C_{BR} ) < m( C )
}

% Step 4: Initialize 

\renewcommand{\partitionings}{
	$
	C \rightarrow
\left(\begin{array}{c I c}
C_{TL} & * \\ \whline
C_{BL} & C_{BR}
\end{array}\right)
	$
	,
	$
	A \rightarrow 
\left(\begin{array}{c}
A_T \\ \whline
A_B
\end{array}\right)
	$
	,
	$
	B \rightarrow 
\left(\begin{array}{c}
B_T \\ \whline
B_B
\end{array}\right)
	$
}

\renewcommand{\partitionsizes}{
	$ C_{BR} $ is $ 0 \times 0 $,
	$ A_B $ and $ B_B $ have $ 0 $ rows
}

% Step 5a: Repartition the operands 

\renewcommand{\repartitionings}{
	$  \left(\begin{array}{c I c}
		C_{TL} & * \\ \whline
		C_{BL} & C_{BR}
		\end{array}\right)
	\rightarrow
	\left(\begin{array}{c c I c}
	C_{00} & * & * \\  
	c_{10}^T & \gamma_{11} & * \\ \whline 
	C_{20} & c_{21} & C_{22}
	\end{array}\right) 
	$
	,
	$  \left(\begin{array}{c}
		A_T \\ \whline
		A_B 
		\end{array}\right) 
		\rightarrow
		\left(\begin{array}{c}
		A_0 \\  
		a_1^T \\ \whline 
		A_2
		\end{array}\right)
		$
	,
	$  \left(\begin{array}{c}
	B_T \\ \whline
	B_B 
	\end{array}\right) 
	\rightarrow
	\left(\begin{array}{c}
	B_0 \\  
	b_1^T \\ \whline 
	B_2
	\end{array}\right)
	$
}

\renewcommand{\repartitionsizes}{
	$ \gamma_{11} $ is $ 1 \times 1 $,
	$ a_1 $ and $ b_1 $ have $ 1 $ row}

% Step 5b: Move the double lines 

\renewcommand{\moveboundaries}{
	$  \left(\begin{array}{c I c}
		C_{TL} & * \\ \whline
		C_{BL} & C_{BR}
		\end{array}\right)
	\leftarrow
	\left(\begin{array}{c I c c}
	C_{00} & * & * \\ \whline 
	c_{10}^T & \gamma_{11} & * \\  
	C_{20} & c_{21} & C_{22}
	\end{array}\right) 
	$
	,
	$  \left(\begin{array}{c}
		A_T \\ \whline
		A_B 
		\end{array}\right) 
		\leftarrow
		\left(\begin{array}{c}
		A_0 \\ \whline 
		a_1^T \\  
		A_2
		\end{array}\right) 
		$
	,
	$  \left(\begin{array}{c}
	B_T \\ \whline
	B_B 
	\end{array}\right) 
	\leftarrow
	\left(\begin{array}{c}
	B_0 \\ \whline 
	b_1^T \\  
	B_2
	\end{array}\right) 
	$
}

% Step 6: State before update
% Note: The below needs editing consistent with loop-invariant!!!

\renewcommand{\beforeupdate}{$ 
	\left(\begin{array}{c c c}
	C_{00} & * & * \\  
	c_{10}^T & \gamma_{11} & * \\ 
	C_{20} & c_{21} & C_{22}
	\end{array}\right) 
	=
	\left(\begin{array}{c c c}
	C_{00} & * & * \\  
	c_{10}^T & \gamma_{11} & * \\ 
	A_2 B_0^T + \widehat C_{20} & A_2 b_1 + \widehat c_{21} & A_2 B_2^T + B_2 A_2^T + \widehat C_{22}
	\end{array}\right) 
$}


% Step 7: State after update
% Note: The below needs editing consistent with loop-invariant!!!

\renewcommand{\afterupdate}{$ 
	\left(\begin{array}{c c c}
	C_{00} & * & * \\  
	c_{10}^T & \gamma_{11} & * \\ 
	C_{20} & c_{21} & C_{22}
	\end{array}\right) 
	=
	\left(\begin{array}{c c c}
	C_{00} & * & * \\  
	a_1^T B_0^T + \widehat c_{10}^T & a_1^T b_1 + b_1^T a_1 + \widehat \gamma_{11} & * \\
	A_2 B_0^T +  \widehat C_{20} & A_2 b_1 + B_2 a_1 + \widehat c_{21} & A_2 B_2^T + B_2 A_2^T + \widehat C_{22}
	\end{array}\right) 
	$}


% Step 8: Insert the updates required to change the 
%         state from that given in Step 6 to that given in Step 7
% Note: The below needs editing!!!

\renewcommand{\update}{
	$
	\begin{array}{l}          % do not delete this line 
	\gamma_{11} := a_1^T b_1 + b_1^T a_1 + \gamma_{11} \\
	c_{10}^T := a_1^T B_0^T + c_{10}^T \\
	c_{21} := B_2 a_1 + c_{21} 
	\end{array}               % do not delete this line 
	$
}








\begin{center}
	\FlaWorksheet
\end{center}

\newpage

\begin{figure}[p]
\begin{center}
	\FlaWorksheetZero
\end{center}
\end{figure}

\begin{figure}[p]
\begin{center}
	\FlaWorksheetOne
\end{center}
\end{figure}

\begin{figure}[p]
\begin{center}
	\FlaWorksheetTwo
\end{center}
\end{figure}

\begin{figure}[p]
\begin{center}
	\FlaWorksheetThree
\end{center}
\end{figure}

\begin{figure}[p]
	\begin{center}
	\FlaWorksheetFour
\end{center}
\end{figure}

\begin{figure}[p]
	\begin{center}
	\FlaWorksheetFive
\end{center}
\end{figure}

\begin{figure}[p]
	\begin{center}
	\FlaWorksheetSix
\end{center}
\end{figure}

\begin{figure}[p]
	\begin{center}
	\FlaWorksheetSeven
\end{center}
\end{figure}

\begin{figure}[p]
	\begin{center}
	\FlaWorksheetEight
\end{center}
\end{figure}

\begin{figure}[p]
	\begin{center}
	\FlaWorksheetNine
\end{center}
\end{figure}

\begin{figure}[p]
\begin{center}
	\FlaAlgorithm
\end{center}
\end{figure}

\end{document}