\documentclass[12pt]{article}

\usepackage{amssymb}
\usepackage{ifthen}
\usepackage[table]{xcolor}
\usepackage{minitoc}
\usepackage{array}

\definecolor{yellow}{cmyk}{0,0,1,0}
\renewcommand{\arraystretch}{1.4}
\newcommand{\R}{\mathbb{R}}

\usepackage{colortbl}

% Page size
\setlength{\oddsidemargin}{-0.5in}
\setlength{\evensidemargin}{-0.5in}
\setlength{\textheight}{10.25in}
\setlength{\textwidth}{7.0in}
\setlength{\topmargin}{-1.35in}

\renewcommand{\arraycolsep}{3pt}


\input color_flatex

\begin{document}
\pagestyle{empty}


\resetsteps % reset all definitions

% Insert output from Spark webpage below


\resetsteps      % Reset all the commands to create a blank worksheet  

% Define the operation to be computed


\resetsteps      % Reset all the commands to create a blank worksheet  

% Define the operation to be computed

\renewcommand{\operation}{ \left[ C \right] := \mbox{\sc syr2k\_ln\_unb\_var4}( A, B, C ) }

\renewcommand{\routinename}{\operation}

% Step 1a: Precondition 

\renewcommand{\precondition}{
	C = \widehat{C}
}

% Step 1b: Postcondition 

\renewcommand{\postcondition}{ 
	\left[C \right]
	=
	\mbox{syr2k\_ln}( A, B, \widehat{C} )
}

% Step 2: Invariant 

\renewcommand{\invariant}{
	\left(\begin{array}{c I c}
	{C_{TL}} & * \\ \whline
	{C_{BL}} & {C_{BR}}
	\end{array}\right) =
	\left(\begin{array}{c I c}
	{A_T B_T^T + B_T A_T^T + \widehat{C}_{TL}} & {*} \\ \whline
	{B_B A_T^T + \widehat{C}_{BL}} & {\widehat{C}_{BR}}
	\end{array}\right)
}

% Step 3: Loop-guard 

\renewcommand{\guard}{
	m( C_{TL} ) < m( C )
}

% Step 4: Initialize 

\renewcommand{\partitionings}{
	$
	C 
	\rightarrow
	\left(\begin{array}{c I c}
	{C_{TL}} & * \\ \whline
	{C_{BL}} & {C_{BR}}
	\end{array}\right)
	$
	,
	$
	A 
	\rightarrow
	\left(\begin{array}{c}
	{ A_T } \\ \whline
	{ A_B }
	\end{array}\right)
	$
	,
	$
	B 
	\rightarrow
	\left(\begin{array}{c}
	{ B_T } \\ \whline
	{ B_B }
	\end{array}\right)
	$
}

\renewcommand{\partitionsizes}{
	$ C_{TL} $ is $ 0 \times 0 $,
	$ A_T $ has $ 0 $ rows,
	$ B_T $ has $ 0 $ rows 
}

% Step 5a: Repartition the operands 

\renewcommand{\repartitionings}{
	$ 
	 \left(\begin{array}{c I c}
	{C_{TL}} & * \\ \whline
	{C_{BL}} & {C_{BR}}
	\end{array}\right)
	\rightarrow
	\left(\begin{array}{c I c c}
	{C_{00}} & * & * \\ \whline
	{c_{10}^T} & {\gamma_{11}} & * \\
	{C_{20}} & {c_{21}} & {C_{22}}
	\end{array}\right)
	$
	,
	$
	\left(\begin{array}{c}
	{ A_T } \\ \whline
	{ A_B }
	\end{array}\right)
	\rightarrow
	\left(\begin{array}{c}
	{A_0} \\ \whline
	{a_1^T} \\
	{A_2}
	\end{array}\right)
	$
	,
	$
	\left(\begin{array}{c}
	{ B_T } \\ \whline
	{ B_B }
	\end{array}\right)
	\rightarrow
	\left(\begin{array}{c}
	{B_0} \\ \whline
	{b_1^T} \\
	{B_2}
	\end{array}\right)
	$
}

\renewcommand{\repartitionsizes}{
	$ \gamma_{11} $ is $ 1 \times 1 $,
	$ a_1 $ has $ 1 $ row,
	$ b_1 $ has $ 1 $ row
}
	

% Step 5b: Move the double lines 

\renewcommand{\moveboundaries}{
		$ 
	\left(\begin{array}{c I c}
	{C_{TL}} & * \\ \whline
	{C_{BL}} & {C_{BR}}
	\end{array}\right)
	\leftarrow
	\left(\begin{array}{c c I c}
	{C_{00}} & * & * \\ 
	{c_{10}^T} & {\gamma_{11}} & * \\ \whline
	{C_{20}} & {c_{21}} & {C_{22}}
	\end{array}\right)
	$
	,
	$
	\left(\begin{array}{c}
	{ A_T } \\ \whline
	{ A_B }
	\end{array}\right)
	\leftarrow
	\left(\begin{array}{c}
	{A_0} \\ 
	{a_1^T} \\ \whline
	{A_2}
	\end{array}\right)
	$
	,
	$
	\left(\begin{array}{c}
	{ B_T } \\ \whline
	{ B_B }
	\end{array}\right)
	\leftarrow
	\left(\begin{array}{c}
	{B_0} \\ 
	{b_1^T} \\ \whline
	{B_2}
	\end{array}\right)
	$
}

% Step 6: State after repartitioning

\renewcommand{\beforeupdate}{
	$  
	\left(\begin{array}{c I c c}
	{C_{00}} & * & * \\ \whline
	{c_{10}^T} & {\gamma_{11}} & * \\ 
	{C_{20}} & {c_{21}} & {C_{22}}
	\end{array}\right)
	=
	\left(\begin{array}{c I c c}
	{A_{0}} {B_{0}^T} + {B_{0}} {A_{0}^T} + \widehat{C}_{00} & 
	* & 
	* \\ \whline
	{b_{1}^T} {A_{0}^T} + \widehat{c}^T_{10}& 
	{\gamma}_{11} &
	* \\ 
	{B_{2}} {A_{0}^T} + \widehat{C}_{20} &
	{c}_{21} & 
	{C}_{22}
	\end{array}\right)
	$
}


% Step 7: State after moving of double lines

\renewcommand{\afterupdate}{
	$  \left(\begin{array}{c c I c}
	{C_{00}} & * & * \\
	{c_{10}^T} & {\gamma_{11}} & * \\ \whline
	{C_{20}} & {c_{21}} & {C_{22}}
	\end{array}\right) 
	= 
	\left(\begin{array}{c c I c}
	{A_{0}} {B_{0}^T} + {B_{0}} {A_{0}^T} + \widehat{C}_{00} & 
	* & 
	* \\ 
	{a_{1}^T} {B_{0}^T} + {b_{1}^T} {A_{0}^T} + \widehat{c}^T_{10}& 
	{a_{1}^T} {b_{1}} + {b_{1}^T} {a_{1}} + \widehat{\gamma}_{11} &
	* \\ \whline
	{B_{2}} {A_{0}^T} + \widehat{C}_{20} &
	{B_{2}} {a_{1}} + \widehat{c}_{21} & 
	{C}_{22}
	\end{array}\right)
	$
}

% Step 8: Insert the updates required to change the 
%         state from that given in Step 6 to that given in Step 7

\renewcommand{\update}{
	$
	\begin{array}{l}
	{\gamma}_{11} := {a_{1}^T} {b_{1}} + {b_{1}^T} {a_{1}} + {\gamma}_{11} \\
	{c}_{10}^T := {a_{1}^T} {B_{0}^T} + {c}^T_{10} \\
	{c}_{21} := {B_{2}} {a_{1}} + {c}_{21}
	\end{array}
	$
}








\begin{center}
	\FlaWorksheet
\end{center}

\newpage

\begin{figure}[p]
\begin{center}
	\FlaWorksheetZero
\end{center}
\end{figure}

\begin{figure}[p]
\begin{center}
	\FlaWorksheetOne
\end{center}
\end{figure}

\begin{figure}[p]
\begin{center}
	\FlaWorksheetTwo
\end{center}
\end{figure}

\begin{figure}[p]
\begin{center}
	\FlaWorksheetThree
\end{center}
\end{figure}

\begin{figure}[p]
	\begin{center}
	\FlaWorksheetFour
\end{center}
\end{figure}

\begin{figure}[p]
	\begin{center}
	\FlaWorksheetFive
\end{center}
\end{figure}

\begin{figure}[p]
	\begin{center}
	\FlaWorksheetSix
\end{center}
\end{figure}

\begin{figure}[p]
	\begin{center}
	\FlaWorksheetSeven
\end{center}
\end{figure}

\begin{figure}[p]
	\begin{center}
	\FlaWorksheetEight
\end{center}
\end{figure}

\begin{figure}[p]
	\begin{center}
	\FlaWorksheetNine
\end{center}
\end{figure}

\begin{figure}[p]
\begin{center}
	\FlaAlgorithm
\end{center}
\end{figure}

\end{document}