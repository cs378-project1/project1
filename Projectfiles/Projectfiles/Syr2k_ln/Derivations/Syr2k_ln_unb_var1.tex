\documentclass[12pt]{article}

\usepackage{amssymb}
\usepackage{ifthen}
\usepackage[table]{xcolor}
\usepackage{minitoc}
\usepackage{array}

\definecolor{yellow}{cmyk}{0,0,1,0}
\renewcommand{\arraystretch}{1.4}
\newcommand{\R}{\mathbb{R}}

\usepackage{colortbl}

% Page size
\setlength{\oddsidemargin}{-0.5in}
\setlength{\evensidemargin}{-0.5in}
\setlength{\textheight}{10.25in}
\setlength{\textwidth}{7.0in}
\setlength{\topmargin}{-1.35in}

\renewcommand{\arraycolsep}{3pt}


\input color_flatex

\begin{document}
\pagestyle{empty}


\resetsteps % reset all definitions

% Insert output from Spark webpage below


\resetsteps      % Reset all the commands to create a blank worksheet  

% Define the operation to be computed

\renewcommand{\operation}{ C = AB^T + BA^T + \widehat{C}}

\renewcommand{\routinename}{\operation}

% Step 1a: Precondition 

\renewcommand{\precondition}{
  C = \widehat{C}
}

% Step 1b: Postcondition 

\renewcommand{\postcondition}{ 
  C = AB^T + BA^T + \widehat{C}
}

% Step 2: Invariant 
% Note: Right-hand side of equalities must be updated appropriately

\renewcommand{\invariant}{
  \left(\begin{array}{c I c}
     C_{TL} & * \\ \whline 
     C_{BL} & C_{BR}
  \end{array}\right) =
  \left(\begin{array}{c I c}
     A_TB_T^T + B_TA_T^T +  \widehat C_{TL}&  * \\ \whline
     \widehat C_{BL} & \widehat C_{BR}
  \end{array}\right)
}

% Step 3: Loop-guard 

\renewcommand{\guard}{
  m( C_{TL} ) < m( C )
}

% Step 4: Initialize 

\renewcommand{\partitionings}{
  $
  A \rightarrow
  \left(\begin{array}{c}
     A_{T} \\ \whline 
     A_{B}
  \end{array}\right)
  $
,
  $
  B \rightarrow
  \left(\begin{array}{c}
     B_{T} \\ \whline 
     B_{B}
  \end{array}\right)
  $
,
  $
  C \rightarrow
  \left(\begin{array}{c I c} 
     C_{TL} & * \\ \whline
     C_{BL} & C_{BR} 
  \end{array}\right) 
  $
}

\renewcommand{\partitionsizes}{
  $ A_T $ has $ 0 $ rows,
  $ B_T $ has $ 0 $ rows,
  $ C_{TL} $ is $ 0 \times 0 $
}

% Step 5a: Repartition the operands 

\renewcommand{\repartitionings}{
  $  \left(\begin{array}{c}
     A_T \\ \whline
     A_B 
  \end{array}\right) 
  \rightarrow
  \left(\begin{array}{c}
     A_0 \\ \whline 
     a_1^T \\  
     A_2
  \end{array}\right)
  $
,
  $  \left(\begin{array}{c}
     B_T \\ \whline
     B_B 
  \end{array}\right) 
  \rightarrow
  \left(\begin{array}{c}
     B_0 \\ \whline 
     b_1^T \\  
     B_2
  \end{array}\right)
  $
,
  $  \left(\begin{array}{c I c}
     C_{TL} & C_{TR} \\ \whline 
     C_{BL} & C_{BR}
  \end{array}\right) 
  \rightarrow
  \left(\begin{array}{c I c c}
     C_{00} & * & * \\ \whline 
     c_{10}^T & \gamma_{11} & * \\  
     C_{20} & c_{21} & C_{22}
  \end{array}\right) 
  $
}

\renewcommand{\repartitionsizes}{
  $ a_1 $ has $ 1 $ row,
  $ b_1 $ has $ 1 $ row,
  $ \gamma_{11} $ is $ 1 \times 1 $}

% Step 5b: Move the double lines 

\renewcommand{\moveboundaries}{
$  \left(\begin{array}{c}
     A_T \\ \whline
     A_B 
  \end{array}\right) 
  \leftarrow
  \left(\begin{array}{c}
     A_0 \\  
     a_1^T \\ \whline 
     A_2
  \end{array}\right) 
  $
,
$  \left(\begin{array}{c}
     B_T \\ \whline
     B_B 
  \end{array}\right) 
  \leftarrow
  \left(\begin{array}{c}
     B_0 \\  
     b_1^T \\ \whline 
     B_2
  \end{array}\right) 
  $
,
$  \left(\begin{array}{c I c}
     C_{TL} & C_{TR} \\ \whline 
     C_{BL} & C_{BR}
  \end{array}\right) 
  \leftarrow
  \left(\begin{array}{c c I c}
     C_{00} & * & * \\  
     c_{10}^T & \gamma_{11} & * \\ \whline 
     C_{20} & c_{21} & C_{22}
  \end{array}\right) 
  $
}

% Step 6: State before update
% Note: The below needs editing consistent with loop-invariant!!!

\renewcommand{\beforeupdate}{$
	\left(\begin{array}{c c c}
	C_{00} & * & * \\  
	c_{10}^T & \gamma_{11} & * \\ 
	C_{20} & c_{21} & C_{22}
	\end{array}\right) =
	\left(\begin{array}{c c c}
	A_0B_0^T + B_0A_0^T + \widehat C_{00} & * & * \\  
	\widehat c_{10}^T & \widehat \gamma_{11} & * \\ 
	\widehat C_{20} & \widehat c_{21} & \widehat C_{22}
	\end{array}\right)
	 $}


% Step 7: State after update
% Note: The below needs editing consistent with loop-invariant!!!

\renewcommand{\afterupdate}{
	$
	\left(\begin{array}{c c c}
	C_{00} & * & * \\  
	c_{10}^T & \gamma_{11} & * \\ 
	C_{20} & c_{21} & C_{22}
	\end{array}\right) =
	\left(\begin{array}{c c c}
	A_0B_0^T + B_0A_0^T + \widehat C_{00} & * & * \\  
	a_1^TB_0^T + b_1^TA_0^T + \widehat  c_{10}^T & a_1^T(b_1^T)^T + b_1^T(a_1^T)^T + \widehat \gamma_{11} & * \\ 
	\widehat C_{20} & \widehat c_{21} & \widehat C_{22}
	\end{array}\right)
 $}


% Step 8: Insert the updates required to change the 
%         state from that given in Step 6 to that given in Step 7
% Note: The below needs editing!!!

\renewcommand{\update}{
$
  \begin{array}{l}          % do not delete this line 
    \mbox{$\gamma_{11} := a_1^T(b_1^T)^T + b_1^T(a_1^T)^T + \gamma_{11}$} \\
    \mbox{$c_{10}^T :=  a_1^TB_0^T + b_1^TA_0^T + c_{10}^T$} % replace \mbox{...} with update line but leave \\
  \end{array}               % do not delete this line 
$
}








\begin{center}
	\FlaWorksheet
\end{center}

\newpage

\begin{figure}[p]
\begin{center}
	\FlaWorksheetZero
\end{center}
\end{figure}

\begin{figure}[p]
\begin{center}
	\FlaWorksheetOne
\end{center}
\end{figure}

\begin{figure}[p]
\begin{center}
	\FlaWorksheetTwo
\end{center}
\end{figure}

\begin{figure}[p]
\begin{center}
	\FlaWorksheetThree
\end{center}
\end{figure}

\begin{figure}[p]
	\begin{center}
	\FlaWorksheetFour
\end{center}
\end{figure}

\begin{figure}[p]
	\begin{center}
	\FlaWorksheetFive
\end{center}
\end{figure}

\begin{figure}[p]
	\begin{center}
	\FlaWorksheetSix
\end{center}
\end{figure}

\begin{figure}[p]
	\begin{center}
	\FlaWorksheetSeven
\end{center}
\end{figure}

\begin{figure}[p]
	\begin{center}
	\FlaWorksheetEight
\end{center}
\end{figure}

\begin{figure}[p]
	\begin{center}
	\FlaWorksheetNine
\end{center}
\end{figure}

\begin{figure}[p]
\begin{center}
	\FlaAlgorithm
\end{center}
\end{figure}

\end{document}